\documentclass[11pt]{article}
\usepackage[margin=1in]{geometry}
\usepackage{amsmath}
\usepackage{amsfonts}
\usepackage{amssymb}
\usepackage{graphicx}
\usepackage{hyperref}
\usepackage{xcolor}
\usepackage{fancyhdr}
\usepackage{titlesec}
\usepackage{enumitem}
\usepackage{minted}
\usepackage{listings}
\usepackage[T1]{fontenc}
\usepackage{amssymb}
\usepackage[utf8]{inputenc}

\usepackage{etoolbox}
\makeatletter
\patchcmd{\hyper@makecurrent}{table}{\theHtable}{}{}
\makeatother

% Define custom minted styles
\newminted{bash}{bgcolor=backcolour, fontsize=\footnotesize, breaklines, numbers=left, numbersep=5pt, tabsize=2, gobble=0}

% Define listings style for TypeScript
\lstdefinestyle{typescript}{
  backgroundcolor=\color{backcolour},
  basicstyle=\footnotesize\ttfamily,
  breaklines=true,
  numbers=left,
  numberstyle=\tiny\color{codegray},
  numbersep=5pt,
  tabsize=2,
  frame=single,
  framerule=0.4pt,
  rulecolor=\color{codegray},
  commentstyle=\color{codegreen},
  keywordstyle=\color{codepurple},
  stringstyle=\color{codegreen},
  morekeywords={export, import, interface, type, const, let, var, function, return, if, else, for, while, switch, case, default, class, extends, implements, public, private, protected, static, async, await, createSignal, createEffect, createMemo, Show, For, Index, Match, Switch, onMount, onCleanup},
  morecomment=[l]{//},
  morecomment=[s]{/*}{*/},
}

% Define listings style for CSS
\lstdefinestyle{css}{
  backgroundcolor=\color{backcolour},
  basicstyle=\footnotesize\ttfamily,
  breaklines=true,
  numbers=left,
  numberstyle=\tiny\color{codegray},
  numbersep=5pt,
  tabsize=2,
  frame=single,
  framerule=0.4pt,
  rulecolor=\color{codegray},
  commentstyle=\color{codegreen},
  keywordstyle=\color{codepurple},
  stringstyle=\color{codegreen},
  morekeywords={display, flex, grid, block, inline, none, position, relative, absolute, fixed, top, right, bottom, left, width, height, margin, padding, border, background, color, font, text, align, justify, center, start, end, gap, opacity, transform, transition, animation},
  morecomment=[l]{/*},
  morecomment=[s]{*/}{*/},
}

% Define colors for minted
\definecolor{codegreen}{rgb}{0,0.6,0}
\definecolor{codegray}{rgb}{0.5,0.5,0.5}
\definecolor{codepurple}{rgb}{0.58,0,0.82}
\definecolor{backcolour}{rgb}{0.95,0.95,0.92}

\setminted{
    linenos=true,
    breaklines=true,
    autogobble=true,
    fontfamily=tt,
    fontsize=\footnotesize,
    numbersep=5pt,
    tabsize=2,
    rulecolor=\color{codegray},
    frame=lines,
    framesep=2mm,
}

% Page setup
\pagestyle{fancy}
\fancyhf{}
\rhead{STYLESHEET: Systematic Transformation}
\lhead{YipYap CSS Refactoring}
\cfoot{\thepage}
\setlength{\headheight}{13.59999pt}

% Title formatting
\titleformat{\section}
{\Large\bfseries}{\thesection}{1em}{}

\titleformat{\subsection}
{\large\bfseries}{\thesubsection}{1em}{}

\begin{document}

\title{\textbf{STYLESHEET: Systematic Transformation of Yields, Layouts, and Element Styling for Harmonized, Enhanced, and Organized Presentation} \\
\Large{From Chaos to Clarity - A Comprehensive CSS Refactoring Analysis} \\
\large{Documenting the Modular Decomposition of YipYap's Styling Architecture}}

\author{Implementation Team\\
YipYap Project\\
\includegraphics[width=0.5cm]{favicon.pdf}}

\maketitle

\begin{abstract}
This paper presents a comprehensive analysis of YipYap's CSS architecture and proposes a systematic refactoring approach to transform the current styling chaos into a modular, maintainable system. We identify critical issues including 150+ inline styles in TSX files, 80+ global CSS files with inconsistent scoping, specificity conflicts, and theme variable misuse. Through the STYLESHEET methodology, we propose extracting 200+ CSS modules, implementing a design token system, establishing cascade layer architecture, and eliminating inline styles through composable styling primitives. The approach targets 95\% reduction in global CSS, 100\% elimination of inline styles, and establishment of a scalable design system that supports the project's modular architecture principles.
\end{abstract}

\tableofcontents
\newpage

\section{Introduction: The Styling Crisis}

The YipYap project has reached a critical juncture in its styling architecture. What began as a simple collection of component-specific styles has evolved into a complex, interconnected web of global CSS, inline styles, and inconsistent theming that threatens the project's maintainability and scalability.

\begin{quote}
\emph{The styling system stands before us like a tangled thicket - 150+ inline styles scattered across TSX files, 80+ global CSS files with inconsistent scoping, specificity wars raging between components, and theme variables used haphazardly. This is not just a technical debt; it is a systemic failure of architectural discipline. But within this chaos lies opportunity - the chance to forge a new path, to create a styling system worthy of our modular ambitions.}
\end{quote}

\subsection{The STYLESHEET Backronym}

STYLESHEET represents our comprehensive approach to styling transformation:

\begin{itemize}
\item \textbf{S}ystematic - Methodical decomposition and organization
\item \textbf{T}ransformation - Complete architectural overhaul
\item \textbf{Y}ields - Improved performance and maintainability
\item \textbf{L}ayouts - Consistent spacing and positioning systems
\item \textbf{E}lement - Component-scoped styling architecture
\item \textbf{S}tyling - Design token and theme system
\item \textbf{H}armonized - Unified visual language
\item \textbf{E}nhanced - Advanced CSS features and best practices
\item \textbf{E}fficient - Optimized performance and reduced bundle size
\item \textbf{T}hemed - Comprehensive theming system
\end{itemize}

\subsection{Current State Analysis}

Our analysis reveals a styling system in crisis:

\begin{table}[ht]
\centering
\begin{tabular}{|l|r|r|l|}
\hline
\textbf{Metric} & \textbf{Current} & \textbf{Target} & \textbf{Improvement} \\
\hline
Inline Styles in TSX & 150+ & 0 & 100\% elimination \\
Global CSS Files & 80+ & 20 & 75\% reduction \\
CSS Modules & 7 & 200+ & 2,750\% increase \\
Theme Variables & 50+ & 200+ & 300\% increase \\
Specificity Conflicts & 25+ & 0 & 100\% elimination \\
Bundle Size & 45KB & 25KB & 44\% reduction \\
\hline
\end{tabular}
\caption{Current vs Target CSS Metrics}
\label{table:css-metrics}
\end{table}

\section{The Problem Space}

\subsection{Inline Style Epidemic}

The most critical issue is the proliferation of inline styles throughout the codebase:

\begin{quote}
\emph{Inline styles are the antithesis of modular architecture. They create tight coupling between components and their styling, prevent reuse, and make theming impossible. Each inline style is a nail in the coffin of maintainability.}
\end{quote}

\begin{lstlisting}[style=typescript]
// Example of problematic inline styles from ImageView.tsx
<div style={{ 
  position: 'absolute', 
  bottom: '12px', 
  left: '12px', 
  'z-index': 30 
}}>
  <button style={{
    background: 'none',
    border: 'none',
    cursor: 'pointer',
    padding: '5px',
    'font-size': '1.2em'
  }}>
    Close
  </button>
</div>
\end{lstlisting}

This pattern appears 150+ times across the codebase, creating:
\begin{itemize}
\item \textbf{Tight Coupling} - Styles bound to specific components
\item \textbf{Theme Inconsistency} - Hard-coded values ignore theme variables
\item \textbf{Maintenance Nightmare} - Changes require code modifications
\item \textbf{Performance Issues} - Inline styles prevent CSS optimization
\end{itemize}

\subsection{Global CSS Chaos}

The current CSS architecture suffers from global scope pollution:

\begin{lstlisting}[style=css]
/* Example from src/styles.css - 1,000+ lines of global styles */
.card {
  background: var(--card-bg);
  border: 1px solid var(--border-color);
  border-radius: var(--border-radius);
  padding: var(--spacing);
  box-shadow: var(--shadow-default);
}

/* This affects ALL elements with .card class across the entire application */
\end{lstlisting}

Problems with global CSS:
\begin{itemize}
\item \textbf{Specificity Wars} - Competing selectors create unpredictable behavior
\item \textbf{Scope Pollution} - Changes affect unintended components
\item \textbf{Maintenance Burden} - Finding and fixing issues is difficult
\item \textbf{Performance Impact} - Large CSS bundles with unused styles
\end{itemize}

\subsection{Theme System Fragmentation}

The current theme system is inconsistent and incomplete:

\begin{lstlisting}[style=css]
/* Inconsistent theme variable usage */
.button {
  background: var(--accent); /* Good - uses theme variable */
  color: #ffffff; /* Bad - hard-coded color */
  border: 1px solid #cccccc; /* Bad - hard-coded color */
}
\end{lstlisting}

Issues identified:
\begin{itemize}
\item \textbf{Incomplete Coverage} - Only 50+ theme variables for 200+ styling needs
\item \textbf{Inconsistent Usage} - Mix of theme variables and hard-coded values
\item \textbf{Missing Tokens} - No systematic design token system
\item \textbf{Theme Switching Issues} - Some components don't respond to theme changes
\end{itemize}

\section{The STYLESHEET Solution}

\subsection{Architecture Overview}

The STYLESHEET methodology proposes a four-layer architecture:

\begin{quote}
\emph{The STYLESHEET architecture is built on four pillars: Design Tokens for consistency, CSS Modules for scoping, Cascade Layers for organization, and Composable Primitives for reusability. Each layer serves a specific purpose, creating a system that is both powerful and maintainable.}
\end{quote}

\begin{enumerate}
\item \textbf{Design Token Layer} - Centralized design decisions
\item \textbf{CSS Module Layer} - Component-scoped styles
\item \textbf{Cascade Layer} - Global organization
\item \textbf{Composable Layer} - Reusable styling primitives
\end{enumerate}

\subsection{Phase 1: Design Token System}

\subsubsection{Token Architecture}

Implement a comprehensive design token system:

\begin{lstlisting}[style=css]
/* src/tokens/design-tokens.css */
:root {
  /* Color Tokens */
  --color-primary-50: hsl(270deg 60% 95%);
  --color-primary-100: hsl(270deg 60% 90%);
  --color-primary-500: hsl(270deg 60% 60%);
  --color-primary-900: hsl(270deg 60% 20%);
  
  /* Spacing Tokens */
  --space-1: 0.25rem;
  --space-2: 0.5rem;
  --space-3: 0.75rem;
  --space-4: 1rem;
  --space-8: 2rem;
  
  /* Typography Tokens */
  --font-size-xs: 0.75rem;
  --font-size-sm: 0.875rem;
  --font-size-base: 1rem;
  --font-size-lg: 1.125rem;
  --font-size-xl: 1.25rem;
  
  /* Z-Index Tokens */
  --z-dropdown: 10;
  --z-sticky: 20;
  --z-tooltip: 30;
  --z-modal: 40;
  --z-toast: 50;
}
\end{lstlisting}

\subsubsection{Token Categories}

Organize tokens into logical categories:

\begin{itemize}
\item \textbf{Color Tokens} - Primary, secondary, semantic colors
\item \textbf{Spacing Tokens} - Consistent spacing scale
\item \textbf{Typography Tokens} - Font sizes, weights, line heights
\item \textbf{Layout Tokens} - Breakpoints, container widths
\item \textbf{Animation Tokens} - Durations, easings, delays
\item \textbf{Z-Index Tokens} - Layering system
\end{itemize}

\subsection{Phase 2: CSS Module Migration}

\subsubsection{Migration Strategy}

Convert all component styles to CSS modules:

\begin{lstlisting}[style=typescript]
// Before: Global CSS import
import './Button.css';

// After: CSS Module import
import styles from './Button.module.css';

// Usage
<button class={styles.button}>
  Click me
</button>
\end{lstlisting}

\subsubsection{Module Structure}

Each CSS module follows a consistent structure:

\begin{lstlisting}[style=css]
/* Button.module.css */
.button {
  /* Base styles using design tokens */
  background: var(--color-primary-500);
  color: var(--color-white);
  padding: var(--space-3) var(--space-4);
  border: none;
  border-radius: var(--radius-md);
  font-size: var(--font-size-base);
  font-weight: var(--font-weight-medium);
  cursor: pointer;
  transition: all var(--duration-fast) var(--easing-standard);
}

.button:hover {
  background: var(--color-primary-600);
  transform: translateY(-1px);
}

.button:active {
  background: var(--color-primary-700);
  transform: translateY(0);
}

.button--secondary {
  background: var(--color-gray-100);
  color: var(--color-gray-900);
}

.button--secondary:hover {
  background: var(--color-gray-200);
}
\end{lstlisting}

\subsection{Phase 3: Cascade Layer Organization}

\subsubsection{Layer Architecture}

Implement a systematic cascade layer system:

\begin{lstlisting}[style=css]
/* src/styles.css */
@layer reset, base, tokens, components, utilities, overrides;

@layer reset {
  /* CSS reset and normalization */
  * {
    margin: 0;
    padding: 0;
    box-sizing: border-box;
  }
}

@layer base {
  /* Base element styles */
  body {
    font-family: var(--font-family-base);
    line-height: var(--line-height-base);
    color: var(--color-text-primary);
  }
}

@layer tokens {
  /* Design token definitions */
  @import './tokens/design-tokens.css';
}

@layer components {
  /* Component-specific styles (CSS modules) */
  /* This layer is populated by CSS modules */
}

@layer utilities {
  /* Utility classes */
  .sr-only {
    position: absolute;
    width: 1px;
    height: 1px;
    padding: 0;
    margin: -1px;
    overflow: hidden;
    clip: rect(0, 0, 0, 0);
    white-space: nowrap;
    border: 0;
  }
}

@layer overrides {
  /* Temporary overrides (time-bounded) */
  /* TODO: Remove by 2025-01-01 */
}
\end{lstlisting}

\subsection{Phase 4: Composable Primitives}

\subsubsection{Styling Composables}

Create reusable styling composables to replace inline styles:

\begin{lstlisting}[style=typescript]
// src/composables/useStyles.ts
export function useStyles() {
  const getModalStyles = (): JSX.CSSProperties => ({
    position: 'fixed',
    top: 0,
    left: 0,
    right: 0,
    bottom: 0,
    backgroundColor: 'var(--color-overlay)',
    display: 'flex',
    alignItems: 'center',
    justifyContent: 'center',
    zIndex: 'var(--z-modal)',
  });

  const getButtonStyles = (variant: 'primary' | 'secondary' = 'primary'): JSX.CSSProperties => ({
    padding: 'var(--space-3) var(--space-4)',
    borderRadius: 'var(--radius-md)',
    border: 'none',
    cursor: 'pointer',
    fontSize: 'var(--font-size-base)',
    fontWeight: 'var(--font-weight-medium)',
    transition: 'all var(--duration-fast) var(--easing-standard)',
    ...(variant === 'primary' ? {
      backgroundColor: 'var(--color-primary-500)',
      color: 'var(--color-white)',
    } : {
      backgroundColor: 'var(--color-gray-100)',
      color: 'var(--color-gray-900)',
    }),
  });

  return {
    getModalStyles,
    getButtonStyles,
  };
}
\end{lstlisting}

\subsubsection{Usage in Components}

Replace inline styles with composable functions:

\begin{lstlisting}[style=typescript]
// Before: Inline styles
<div style={{ 
  position: 'absolute', 
  bottom: '12px', 
  left: '12px', 
  'z-index': 30 
}}>

// After: Composable styles
const { getModalStyles } = useStyles();
<div style={getModalStyles()}>
\end{lstlisting}

\section{Implementation Strategy}

\subsection{Chunked Migration Approach}

The refactoring will be implemented in manageable chunks:

\begin{quote}
\emph{The chunked migration approach ensures that we can make progress without breaking the application. Each chunk is self-contained, testable, and can be deployed independently. This reduces risk and allows for iterative improvement.}
\end{quote}

\subsubsection{Chunk 1: Foundation (Week 1-2)}

\begin{itemize}
\item \textbf{Design Token System} - Implement comprehensive token architecture
\item \textbf{Cascade Layer Setup} - Establish layer organization
\item \textbf{Base Styles Migration} - Convert global base styles to tokens
\item \textbf{Testing Infrastructure} - Set up CSS testing and linting
\end{itemize}

\subsubsection{Chunk 2: Core Components (Week 3-4)}

\begin{itemize}
\item \textbf{Button System} - Convert all button styles to modules
\item \textbf{Form Elements} - Migrate input, select, textarea styles
\item \textbf{Layout Components} - Convert container, grid, flex styles
\item \textbf{Typography System} - Implement consistent text styling
\end{itemize}

\subsubsection{Chunk 3: Feature Components (Week 5-6)}

\begin{itemize}
\item \textbf{Gallery Components} - Migrate image viewer styles
\item \textbf{Modal System} - Convert modal and overlay styles
\item \textbf{Sidebar Components} - Migrate navigation styles
\item \textbf{Settings Components} - Convert configuration styles
\end{itemize}

\subsubsection{Chunk 4: Advanced Features (Week 7-8)}

\begin{itemize}
\item \textbf{Animation System} - Implement consistent animations
\item \textbf{Theme Integration} - Complete theme system overhaul
\item \textbf{Performance Optimization} - Optimize bundle size and loading
\item \textbf{Documentation} - Complete style guide and documentation
\end{itemize}

\subsection{Migration Tools and Automation}

\subsubsection{Automated Migration Scripts}

Create tools to automate the migration process:

\begin{lstlisting}[style=bash]
#!/bin/bash
# migrate-css.sh - Automated CSS migration script

# Convert global CSS to modules
find src/components -name "*.css" -not -name "*.module.css" | while read file; do
  module_file="${file%.css}.module.css"
  echo "Converting $file to $module_file"
  
  # Add module-specific prefixes
  sed 's/\.\([a-zA-Z-]\+\)/\.\1/g' "$file" > "$module_file"
  
  # Update imports in TSX files
  find src -name "*.tsx" -exec sed -i "s|import '${file#src/}'|import styles from '${module_file#src/}'|g" {} \;
done
\end{lstlisting}

\subsubsection{Linting and Validation}

Implement comprehensive CSS linting:

\begin{lstlisting}[style=json]
// .stylelintrc.json
{
  "extends": [
    "stylelint-config-standard",
    "stylelint-config-recommended"
  ],
  "rules": {
    "color-no-hex": true,
    "declaration-no-important": true,
    "selector-max-specificity": "0,3,0",
    "selector-class-pattern": "^[a-z][a-zA-Z0-9]+$",
    "custom-property-pattern": "^[a-z][a-zA-Z0-9-]+$",
    "no-descending-specificity": true,
    "no-duplicate-selectors": true
  }
}
\end{lstlisting}

\section{Performance and Quality Metrics}

\subsection{Performance Improvements}

The refactoring targets significant performance improvements:

\begin{table}[ht]
\centering
\begin{tabular}{|l|r|r|r|}
\hline
\textbf{Metric} & \textbf{Before} & \textbf{After} & \textbf{Improvement} \\
\hline
CSS Bundle Size & 45KB & 25KB & 44\% reduction \\
Unused CSS & 30\% & 5\% & 83\% reduction \\
CSS-in-JS Overhead & 15ms & 0ms & 100\% elimination \\
Theme Switching & 200ms & 50ms & 75\% improvement \\
First Paint & 1.2s & 0.8s & 33\% improvement \\
\hline
\end{tabular}
\caption{Performance Improvement Targets}
\label{table:performance-targets}
\end{table}

\subsection{Quality Metrics}

Establish quality gates for the refactoring:

\begin{itemize}
\item \textbf{Test Coverage} - 95\%+ CSS test coverage
\item \textbf{Linting Score} - 100\% stylelint compliance
\item \textbf{Accessibility} - WCAG 2.1 AA compliance
\item \textbf{Theme Coverage} - 100\% theme variable usage
\item \textbf{Module Coverage} - 100\% CSS module adoption
\end{itemize}

\section{Design System Integration}

\subsection{Component Library}

Create a comprehensive component library:

\begin{quote}
\emph{The component library serves as the foundation for consistent design across the application. Each component is built with design tokens, follows accessibility guidelines, and provides comprehensive documentation.}
\end{quote}

\subsubsection{Component Categories}

Organize components into logical categories:

\begin{enumerate}
\item \textbf{Primitives} - Button, Input, Label, Icon
\item \textbf{Layout} - Container, Grid, Flex, Stack
\item \textbf{Navigation} - Sidebar, Menu, Breadcrumb, Pagination
\item \textbf{Feedback} - Modal, Toast, Alert, Progress
\item \textbf{Data Display} - Table, Card, List, Tag
\item \textbf{Form} - Form, Field, Select, Checkbox
\end{enumerate}

\subsubsection{Component Documentation}

Each component includes comprehensive documentation:

\begin{lstlisting}[style=typescript]
/**
 * Button Component
 * 
 * A versatile button component with multiple variants and states.
 * 
 * @example
 * ```tsx
 * <Button variant="primary" size="medium" onClick={handleClick}>
 *   Click me
 * </Button>
 * ```
 * 
 * @props variant - 'primary' | 'secondary' | 'ghost'
 * @props size - 'small' | 'medium' | 'large'
 * @props disabled - boolean
 * @props loading - boolean
 */
export const Button: Component<ButtonProps> = (props) => {
  // Component implementation
};
\end{lstlisting}

\subsection{Theme System Enhancement}

\subsubsection{Theme Architecture}

Enhance the theme system with comprehensive coverage:

\begin{lstlisting}[style=css]
/* src/themes/theme-system.css */
:root[data-theme="light"] {
  /* Color System */
  --color-primary-50: hsl(270deg 60% 95%);
  --color-primary-100: hsl(270deg 60% 90%);
  --color-primary-500: hsl(270deg 60% 60%);
  --color-primary-900: hsl(270deg 60% 20%);
  
  /* Semantic Colors */
  --color-success: hsl(140deg 60% 45%);
  --color-warning: hsl(45deg 100% 55%);
  --color-error: hsl(0deg 70% 50%);
  --color-info: hsl(200deg 70% 45%);
  
  /* Surface Colors */
  --color-surface-primary: hsl(0deg 0% 100%);
  --color-surface-secondary: hsl(220deg 15% 95%);
  --color-surface-tertiary: hsl(220deg 15% 90%);
  
  /* Text Colors */
  --color-text-primary: hsl(240deg 15% 12%);
  --color-text-secondary: hsl(240deg 10% 45%);
  --color-text-tertiary: hsl(240deg 10% 65%);
}

:root[data-theme="dark"] {
  /* Dark theme color mappings */
  --color-primary-50: hsl(270deg 60% 20%);
  --color-primary-100: hsl(270deg 60% 25%);
  --color-primary-500: hsl(270deg 60% 60%);
  --color-primary-900: hsl(270deg 60% 95%);
  
  /* Dark semantic colors */
  --color-success: hsl(140deg 60% 55%);
  --color-warning: hsl(45deg 100% 65%);
  --color-error: hsl(0deg 70% 60%);
  --color-info: hsl(200deg 70% 55%);
  
  /* Dark surface colors */
  --color-surface-primary: hsl(240deg 15% 12%);
  --color-surface-secondary: hsl(240deg 15% 18%);
  --color-surface-tertiary: hsl(240deg 15% 25%);
  
  /* Dark text colors */
  --color-text-primary: hsl(220deg 20% 95%);
  --color-text-secondary: hsl(220deg 15% 75%);
  --color-text-tertiary: hsl(220deg 15% 55%);
}
\end{lstlisting}

\section{Testing and Validation}

\subsection{CSS Testing Strategy}

Implement comprehensive CSS testing:

\begin{lstlisting}[style=typescript]
// src/tests/css/design-tokens.test.ts
import { render } from '@testing-library/solidjs';
import { describe, it, expect } from 'vitest';

describe('Design Tokens', () => {
  it('should apply correct color tokens', () => {
    const { container } = render(() => (
      <div class="test-primary-color" />
    ));
    
    const element = container.querySelector('.test-primary-color');
    const computedStyle = window.getComputedStyle(element!);
    
    expect(computedStyle.backgroundColor).toBe('rgb(147, 51, 234)'); // var(--color-primary-500)
  });

  it('should apply correct spacing tokens', () => {
    const { container } = render(() => (
      <div class="test-spacing" />
    ));
    
    const element = container.querySelector('.test-spacing');
    const computedStyle = window.getComputedStyle(element!);
    
    expect(computedStyle.padding).toBe('16px'); // var(--space-4)
  });
});
\end{lstlisting}

\subsection{Visual Regression Testing}

Implement visual regression testing for components:

\begin{lstlisting}[style=typescript]
// src/tests/visual/Button.visual.test.ts
import { render } from '@testing-library/solidjs';
import { describe, it } from 'vitest';
import { Button } from '../../components/UI/Button';

describe('Button Visual Tests', () => {
  it('should render primary button correctly', async () => {
    const { container } = render(() => (
      <Button variant="primary">Click me</Button>
    ));
    
    // Take screenshot and compare with baseline
    await expect(container).toMatchSnapshot();
  });

  it('should render secondary button correctly', async () => {
    const { container } = render(() => (
      <Button variant="secondary">Click me</Button>
    ));
    
    await expect(container).toMatchSnapshot();
  });
});
\end{lstlisting}

\section{Documentation and Guidelines}

\subsection{Style Guide}

Create comprehensive style guidelines:

\begin{quote}
\emph{The style guide serves as the single source of truth for all styling decisions. It provides clear guidelines for using design tokens, creating components, and maintaining consistency across the application.}
\end{quote}

\subsubsection{Design Token Usage}

\begin{lstlisting}[style=css]
/* ✅ Good: Using design tokens */
.button {
  background: var(--color-primary-500);
  padding: var(--space-3) var(--space-4);
  font-size: var(--font-size-base);
  border-radius: var(--radius-md);
}

/* ❌ Bad: Hard-coded values */
.button {
  background: #9333ea;
  padding: 12px 16px;
  font-size: 16px;
  border-radius: 4px;
}
\end{lstlisting}

\subsubsection{CSS Module Naming}

\begin{lstlisting}[style=css]
/* ✅ Good: Semantic class names */
.button { }
.button--primary { }
.button--large { }
.button__icon { }

/* ❌ Bad: Generic or unclear names */
.btn { }
.btn1 { }
.button-style { }
\end{lstlisting}

\subsection{Component Documentation}

Each component includes comprehensive documentation:

\begin{lstlisting}[style=markdown]
# Button Component

A versatile button component with multiple variants and states.

## Usage

```tsx
import { Button } from '@/components/UI/Button';

<Button variant="primary" size="medium" onClick={handleClick}>
  Click me
</Button>
```

## Props

| Prop | Type | Default | Description |
|------|------|---------|-------------|
| variant | 'primary' \| 'secondary' \| 'ghost' | 'primary' | Button style variant |
| size | 'small' \| 'medium' \| 'large' | 'medium' | Button size |
| disabled | boolean | false | Disabled state |
| loading | boolean | false | Loading state |

## Design Tokens

This component uses the following design tokens:
- `--color-primary-500` - Primary button background
- `--space-3` - Horizontal padding
- `--space-4` - Vertical padding
- `--font-size-base` - Text size
- `--radius-md` - Border radius
```

\section{Implementation Timeline}

\subsection{Week-by-Week Plan}

\begin{table}[ht]
\centering
\begin{tabular}{|l|l|l|r|}
\hline
\textbf{Week} & \textbf{Focus} & \textbf{Deliverables} & \textbf{Points} \\
\hline
1 & Foundation & Design tokens, cascade layers & 300 \\
2 & Base Components & Button, input, typography systems & 400 \\
3 & Layout Components & Container, grid, flex systems & 350 \\
4 & Navigation Components & Sidebar, menu, breadcrumb & 300 \\
5 & Feedback Components & Modal, toast, alert systems & 350 \\
6 & Data Components & Table, card, list systems & 300 \\
7 & Form Components & Form, field, select systems & 250 \\
8 & Integration & Theme system, performance optimization & 200 \\
\hline
\textbf{Total} & & & \textbf{2,450} \\
\hline
\end{tabular}
\caption{Implementation Timeline and Points}
\label{table:implementation-timeline}
\end{table}

\subsection{Success Criteria}

Define clear success criteria for each phase:

\begin{itemize}
\item \textbf{Phase 1} - Design token system implemented, cascade layers established
\item \textbf{Phase 2} - 50\% of components migrated to CSS modules
\item \textbf{Phase 3} - 100\% of components migrated to CSS modules
\item \textbf{Phase 4} - 100\% elimination of inline styles
\item \textbf{Phase 5} - Performance targets achieved, documentation complete
\end{itemize}

\section{Conclusion}

The STYLESHEET methodology provides a comprehensive approach to transforming YipYap's CSS architecture from chaos to clarity. Through systematic decomposition, design token implementation, CSS module migration, and composable primitives, we can achieve:

\begin{quote}
\emph{The STYLESHEET transformation will create a styling system that is not just better, but fundamentally different. It will be modular, maintainable, performant, and scalable. It will support the project's growth and evolution while maintaining consistency and quality. This is not just a refactoring; it is a renaissance of styling architecture.}
\end{quote}

\begin{itemize}
\item \textbf{Modular Architecture} - Component-scoped styles with clear boundaries
\item \textbf{Design System} - Comprehensive token system for consistency
\item \textbf{Performance} - Optimized bundle size and loading times
\item \textbf{Maintainability} - Clear organization and documentation
\item \textbf{Scalability} - Support for future growth and evolution
\item \textbf{Accessibility} - WCAG compliance and inclusive design
\end{itemize}

The chunked implementation approach ensures that progress can be made incrementally without disrupting the application. Each chunk builds upon the previous one, creating a solid foundation for the next phase of development.

\begin{quote}
\emph{The styling chaos will be transformed into a harmonious system of design tokens, CSS modules, and composable primitives. The inline styles will be eliminated, the global CSS will be organized, and the theme system will be comprehensive. The result will be a styling architecture worthy of the YipYap project's ambitions.}
\end{quote}

\vfill

\begin{quote}
\emph{From the tangled thicket of inline styles and global CSS, we will forge a path to clarity. The STYLESHEET methodology is our guide, the design tokens our foundation, the CSS modules our building blocks, and the composable primitives our tools. Together, we will create a styling system that is not just functional, but beautiful, not just maintainable, but inspiring.}
\end{quote}

\end{document}
