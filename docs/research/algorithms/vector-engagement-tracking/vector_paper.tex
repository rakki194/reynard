\documentclass[10pt]{article}

\usepackage{graphicx}
\usepackage{amsmath}
\usepackage{algorithm}
\usepackage{algpseudocode}
\usepackage{tikz}
\usepackage{hyperref}
\usepackage{booktabs}
\usepackage{listings}
\usepackage{float}
\usepackage{microtype}
\lstset{
    basicstyle=\small\ttfamily,
    breaklines=true,
    breakatwhitespace=true,
    showstringspaces=false,
    columns=flexible
}

\begin{document}

\title{VECTOR: Vectorized Engagement and COnsumption Tracking for Optimal Reporting}

\author{Technical Documentation Team\\
Reynard Project\\
\includegraphics[width=0.5cm]{../../shared-assets/favicon.pdf}}

\maketitle

\begin{abstract}
We propose VECTOR (Vectorized Engagement and COnsumption Tracking for Optimal Reporting), a novel system designed for real-time user engagement and usage tracking within interactive applications. VECTOR leverages advanced linear algebra techniques, including Principal Component Analysis (PCA) and Singular Value Decomposition (SVD), to transform raw user interaction data into meaningful, actionable insights. Our system enables the identification of key engagement patterns, user segmentation, and predictive analytics for personalized experiences and optimized feature development, achieving efficient processing through optimized matrix operations and incremental updates.
\end{abstract}

\section{Introduction}
Understanding user behavior and engagement is critical for the success of modern interactive applications. Traditional analytics often provide aggregated metrics, lacking the granularity needed to identify nuanced user interactions or predict future behaviors. VECTOR addresses these limitations by introducing a sophisticated linear algebra-based approach to analyze complex usage patterns, providing deeper insights and enabling more personalized and responsive application experiences.

\section{System Architecture}
\subsection{Core Components}
VECTOR consists of five main subsystems:
\begin{enumerate}
    \item Data Ingestion
    \item Data Vectorization
    \item Linear Algebra Module
    \item Pattern Recognition and Engagement Scoring
    \item Reporting and Visualization
\end{enumerate}

\subsection{Mathematical Foundation}
The core of VECTOR's analytical power lies in its application of linear algebra to model user behavior. User interactions are transformed into a dense, high-dimensional matrix, where rows represent users and columns represent various engagement metrics or features (e.g., time spent on a feature, clicks, frequency of use).

Let $M$ be the user-feature matrix of size $U \times F$, where $U$ is the number of users and $F$ is the number of features/metrics.
$M_{ij}$ represents the value of feature $j$ for user $i$.

\subsubsection{Principal Component Analysis (PCA)}
PCA is used for dimensionality reduction and to identify the most significant underlying engagement patterns.
The covariance matrix $C = \frac{1}{U-1} M^T M$ is computed.
Eigenvalues and eigenvectors of $C$ are then extracted. The principal components are linear combinations of the original features that capture the maximum variance in the data.
\[
    M_{reduced} = M W
\]
where $W$ is the transformation matrix composed of the top $k$ eigenvectors.

\subsubsection{Singular Value Decomposition (SVD)}
SVD is employed for latent factor analysis, identifying hidden patterns and relationships between users and features. It can also be used for collaborative filtering and anomaly detection.
\[
    M = U \Sigma V^T
\]
where $U$ and $V$ are orthogonal matrices, and $\Sigma$ is a diagonal matrix containing singular values. The singular values indicate the strength of the latent factors.

This linear algebra framework provides a robust and scalable method for dissecting complex user interaction data.

\section{Algorithmic Implementation}
\subsection{Data Vectorization}
Raw user event streams (e.g., clicks, views, session durations) are preprocessed and transformed into numerical vectors. Categorical data is one-hot encoded, and continuous data is normalized. These vectors are then aggregated over defined time windows (e.g., daily, weekly) to form the user-feature matrix $M$.

\subsection{Linear Algebra Algorithms}
The preprocessed user-feature matrix is fed into the Linear Algebra Module.
\begin{algorithmic}[1]
\Function{ApplyPCA}{Matrix M, k}
    \State Normalize M
    \State Compute CovarianceMatrix = M.T @ M
    \State Eigenvalues, Eigenvectors = Eig(CovarianceMatrix)
    \State Sort Eigenvectors by Eigenvalues (descending)
    \State Select top k Eigenvectors to form TransformationMatrix W
    \State Return M @ W
\EndFunction
\Statex
\Function{ApplySVD}{Matrix M}
    \State U, Sigma, VT = SVD(M)
    \State Return U, Sigma, VT
\EndFunction
\end{algorithmic}

\subsection{Pattern Recognition and Engagement Scoring}
Insights are extracted from the reduced-dimension matrices.
-   	extbf{User Segmentation}: Clustering algorithms (e.g., k-means) are applied to the PCA-transformed data to group users with similar engagement patterns.
-   	extbf{Engagement Scoring}: A composite engagement score can be derived for each user by weighting different principal components or latent factors identified through SVD.
-   	extbf{Anomaly Detection}: Deviations from learned patterns using SVD can signal unusual or problematic user behaviors.

\section{Performance Optimizations}
To ensure real-time or near real-time processing of user data, VECTOR incorporates several performance enhancements:
\begin{itemize}
    \item 	extbf{Batch Processing}: Data ingestion and initial matrix construction are performed in batches to minimize computational overhead.
    \item 	extbf{Incremental Updates}: For large datasets, incremental PCA or incremental SVD algorithms are utilized to update the models without recomputing from scratch. This reduces computation for new data points.
    \item 	extbf{Sparse Matrix Representation}: For matrices with many zero entries (common in usage data), sparse matrix data structures are used to conserve memory and speed up operations.
    \item 	extbf{Optimized Linear Algebra Libraries}: Leverage highly optimized numerical libraries (e.g., NumPy, SciPy) for efficient matrix operations.
\end{itemize}

\section{User Engagement and Usage Tracking}
VECTOR provides a comprehensive framework for understanding and acting upon user engagement and usage patterns:
\begin{itemize}
    \item 	extbf{Feature Adoption Analysis}: Identify which features are most used and by which user segments.
    \item 	extbf{Session Analysis}: Gain insights into session duration, frequency, and common navigation paths.
    \item 	extbf{Churn Prediction}: Develop predictive models based on changes in engagement patterns to identify users at risk of churning.
    \item 	extbf{Personalization}: Use identified latent factors to tailor content, recommendations, or UI elements to individual user preferences.
    \item 	extbf{A/B Testing Insights}: Analyze engagement differences between experiment groups with greater statistical power.
\end{itemize}

\section{VECTOR Integration in YipYap}
In the context of the YipYap platform, VECTOR would integrate seamlessly to provide insights into how users interact with image annotation tools, gallery features, and caption generation capabilities.
The data flow would involve:
\begin{enumerate}
    \item 	extbf{Event Capture}: Frontend components emit discrete events (e.g., 'image.view', 'tag.add', 'caption.generate').
    \item 	extbf{Data Queueing}: Events are asynchronously queued and sent to a backend ingestion service.
    \item 	extbf{Backend Processing}: The backend aggregates events, constructs the user-feature matrix, and periodically runs the Linear Algebra Module.
    \item 	extbf{Insight Dissemination}: Processed insights (e.g., user segments, engagement scores) are stored and made accessible for reporting dashboards or internal services (e.g., for personalized feature exposure).
\end{enumerate}

\section{Key Achievements}
The implementation of VECTOR is expected to yield several significant improvements:
\begin{itemize}
    \item Deep, data-driven understanding of user behavior beyond simple metrics.
    \item The ability to proactively identify and address user pain points or opportunities.
    \item Enhanced personalization leading to improved user satisfaction and retention.
    \item Optimized resource allocation for feature development based on actual usage patterns.
    \item Robust anomaly detection for security or usability issues.
\end{itemize}

\section{Conclusion}
VECTOR proposes a powerful, linear algebra-driven approach to user engagement and usage tracking. By transforming complex interaction data into a structured mathematical problem, VECTOR provides unparalleled insights into user behavior, enabling data-driven decision-making for product optimization and personalized user experiences. Our contributions include a novel application of PCA and SVD for engagement analysis, coupled with robust performance optimizations, demonstrating its potential for real-world interactive applications like YipYap.

\bibliographystyle{IEEEtran}
\begin{thebibliography}{99}
% Placeholder for references. Examples would include papers on PCA, SVD,
% user behavior modeling, and data analytics.
\bibitem{jolliffe2002pca} I. T. Jolliffe, "Principal Component Analysis," Springer, 2002.
\bibitem{golub1996matrix} G. H. Golub, C. F. Van Loan, "Matrix Computations," Johns Hopkins University Press, 1996.
\bibitem{resnick1994grouplens} P. Resnick, N. Iacovou, M. Suchak, P. Bergstrom, J. Riedl, "GroupLens: An Open Architecture for Collaborative Filtering of Netnews," In Proceedings of CSCW, 1994.
\end{thebibliography}

\end{document} 