\documentclass[10pt]{article}

\usepackage{graphicx}
\usepackage{amsmath}
\usepackage{algorithm}
\usepackage{algpseudocode}
\usepackage{tikz}
\usepackage{hyperref}
\usepackage{booktabs}
\usepackage{listings}
\usepackage{microtype}
\usepackage{caption}
\usepackage{fancyvrb}
\usepackage{geometry}

% Set page margins to reduce overfull boxes
\geometry{
  margin=1.25in,
  includeheadfoot
}

% Adjust float parameters to allow larger floats
\renewcommand{\topfraction}{.9}
\renewcommand{\bottomfraction}{.9}
\renewcommand{\textfraction}{.1}
\renewcommand{\floatpagefraction}{.8}

% Configure listings for better code formatting
\lstset{
    basicstyle=\small\ttfamily,
    breaklines=true,
    breakatwhitespace=true,
    showstringspaces=false,
    columns=flexible,
    postbreak=\mbox{\textcolor{red}{$\hookrightarrow$}\space},
    breakindent=0pt
}

% Adjust paragraph spacing
\setlength{\parskip}{6pt}
\setlength{\parindent}{0pt}

% Adjust line spacing
\linespread{1.1}

% Adjust code block margins
\fvset{
    xleftmargin=1em,
    xrightmargin=1em
}

\begin{document}

\title{NAVIGATE: Navigation and View Integration for Guided User Experience}

\author{Technical Documentation Team\\
Reynard Project\\
\includegraphics[width=0.5cm]{../../shared-assets/favicon.pdf}}

\maketitle

\begin{abstract}
This paper introduces the NAVIGATE system, a proposed redesign initiative for enhancing user interface responsiveness and navigation fluidity within the YipYap application. We detail the current implementation of the settings tab view switching mechanism, including its animation retrigger protection, as a foundational example of current UI interaction management.
\end{abstract}

\section{Introduction}

Modern user interfaces demand highly responsive and intuitive navigation experiences. The YipYap application, with its rich set of features and settings, relies on efficient UI transitions to provide a seamless user experience. This document outlines the proposed NAVIGATE system, a comprehensive redesign effort aimed at further enhancing UI responsiveness and user interaction dynamics. As a starting point, we will detail the existing settings tab view switching mechanism, highlighting its current implementation and the recent improvements for animation retrigger protection.

\section{Settings Tab View Switching Mechanism}

The settings panel in YipYap (implemented in \texttt{src/components/Settings/Settings.tsx}) provides a tabbed interface for users to navigate between various configuration sections. The core of this functionality lies within the \texttt{switchView} function, which manages the active settings view and orchestrates the associated visual transitions.

\subsection{Core Components and State}

The view switching mechanism utilizes several SolidJS signals to manage its state:
\begin{itemize}
    \item \texttt{activeView}: A signal that holds the identifier of the currently active settings tab (e.g., \texttt{'main'}, \texttt{'transformations'}, \texttt{'experimental'}).
    \item \texttt{isTransitioning}: A boolean signal that indicates whether a view transition animation is currently in progress. This is crucial for preventing new transitions from starting before the current one completes.
    \item \texttt{lastViewChangeTime}: A numerical signal that stores the \texttt{Date.now()} timestamp of the last time a view change was initiated. This is used for the retrigger protection mechanism.
\end{itemize}

\subsection{The \texttt{switchView} Function}

The \texttt{switchView} function is triggered when a user clicks on one of the settings tab buttons. Its logic ensures a controlled and smooth transition between views, incorporating a retrigger protection mechanism.

\begin{enumerate}
    \item \textbf{Debounce Check (Retrigger Protection):}
    Upon invocation, the function first checks if a transition is already in progress (\texttt{isTransitioning()}) AND if the time elapsed since the \texttt{lastViewChangeTime()} is less than a predefined \texttt{DEBOUNCE\_TIME} (currently 500ms). If both conditions are true, the function immediately returns, preventing rapid, successive clicks from re-triggering the animation and causing UI jank or unexpected behavior. This ensures that the user's intent to switch views is respected, but with a brief delay to allow ongoing animations to complete, providing a more robust and visually consistent experience.

    \item \textbf{Same View Check:}
    If the requested \texttt{view} is already the \texttt{activeView()}, and it's not the 'main' view, the system initiates a transition back to the 'main' view after a 300ms delay. This provides a toggle-like behavior for sub-settings panels, allowing users to quickly return to the main settings section.

    \item \textbf{Transition Initiation:}
    If a new view is being selected, the \texttt{isTransitioning} signal is set to \texttt{true}, and the \texttt{lastViewChangeTime} is updated to the current timestamp (\texttt{Date.now()}). A \texttt{setTimeout} is then used to delay the actual \texttt{activeView} update and the resetting of \texttt{isTransitioning} to \texttt{false}. The current delay is set to 300ms, which corresponds to the duration of the CSS transition animations defined in \texttt{src/components/Settings/Settings.css}. This delay ensures that the visual transition has time to complete before the new content is fully rendered and interaction is re-enabled.
\end{enumerate}

\subsection{Visual Transitions}

The visual transitions between different settings views are primarily handled by CSS. The \texttt{settings-content} div, which wraps each view, has a \texttt{classList} bound to the \texttt{isTransitioning()} signal. When \texttt{isTransitioning} is \texttt{true}, a \texttt{transitioning} class is applied, which typically triggers a CSS transition (e.g., opacity, transform) to create a smooth animation effect.

\subsection{Current Code Snippet}

The relevant part of the \texttt{src/components/Settings/Settings.tsx} file demonstrating this logic is:

\begin{Verbatim}[frame=lines, numbers=left, tabsize=2, fontsize=\small, xleftmargin=1em, xrightmargin=1em, breaklines=true]
// ... existing code ...
const DEBOUNCE_TIME = 500; // milliseconds

export const Settings: Component<{ onClose: () => void }> = (props) => {
  const app = useAppContext();
  const escape = useGlobalEscapeManager();
  const [activeView, setActiveView] = createSignal<'main' | 'help' | 'transformations' | 'experimental' | 'tagger' | 'captioners' | 'captionTypes' | 'tagsAndSuggestions' | 'boundingBox' | 'performance' | 'danger' | 'versionControl' | 'assistant'>('main');
  const [isTransitioning, setIsTransitioning] = createSignal(false);
  const [hoveredView, setHoveredView] = createSignal<string | null>(null);
  const t = useTranslations();
  const [lastViewChangeTime, setLastViewChangeTime] = createSignal(0);

  // RTL languages
  const rtlLanguages = ['he', 'ar', 'fa'];
  const isRtl = () => rtlLanguages.includes(app.locale);

  onMount(() => {
    escape.setOverlayState("settings", true);
    const unregister = escape.registerHandler("settings", props.onClose);

    onCleanup(() => {
      escape.setOverlayState("settings", false);
      unregister();
    });
  });

  const switchView = (view: 'main' | 'help' | 'transformations' | 'experimental' | 'tagger' | 'captioners' | 'captionTypes' | 'tagsAndSuggestions' | 'boundingBox' | 'performance' | 'danger' | 'versionControl' | 'assistant') => {
    const now = Date.now();
    if (isTransitioning() && now - lastViewChangeTime() < DEBOUNCE_TIME) {
      return; // Prevent re-triggering too quickly
    }

    if (activeView() === view) {
      if (view !== 'main') {
        setIsTransitioning(true);
        setLastViewChangeTime(now);
        setTimeout(() => {
          setActiveView('main');
          setIsTransitioning(false);
        }, 300);
      }
      return;
    }

    setIsTransitioning(true);
    setLastViewChangeTime(now);
    setTimeout(() => {
      setActiveView(view);
      setIsTransitioning(false);
    }, 300);
  };
// ... existing code ...
\end{Verbatim}

This setup ensures that settings tab transitions are visually appealing and robust against rapid user interactions.

\end{document} 